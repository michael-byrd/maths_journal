\documentclass[11pt,letterpaper]{tufte-book}
\usepackage[latin1]{inputenc}
\usepackage{amsmath, amsfonts, amssymb}
\usepackage{graphicx}


\usepackage{subfiles}
\usepackage{import}

%%% Reset Chapter Count for each Part

\makeatletter
\@addtoreset{chapter}{part}
\makeatother  


%%% Math Environments %%%
\usepackage{amsthm}
\theoremstyle{plain}
\newtheorem{theorem}{Theorem}
\newtheorem{lemma}{Lemma}
\newtheorem{corollary}{Corollary}
\newtheorem{proposition}{Proposition}
\newtheorem{conjecture}{Conjecture}
\theoremstyle{remark}
\newtheorem*{remark}{Remark}
\theoremstyle{definition}
\newtheorem{definition}{Definition}
\newtheorem{example}{Example}

%%%%%%%%%%%%%%%%%%%%%%%%%

% When creating a new book.
% Create the book directory, tex, chapter, and section file to get things started
% GitHub does not push empty directories.

%%% input commands %%%

% input parts
% \input{books/SUBJECTDIR/SUBJECT.tex}

% input chapters
% \input{books/SUBJECTDIR/CHAPTRT.tex}

% input sections
% \input{books/SUBJECTDIR/sections/SECTION.tex}

%%%%%%%%%%%%%%%%%%%%%%



\title{Mathematics Journal}
\author{Michael C.R. Byrd Jr.}

\begin{document}
	
	\maketitle
	
	\tableofcontents	

%	This is the main file. Each Part of this text will be a subject of Mathematics. These parts will serve as standalone self-contained books.
	
%	Parts will be defined in this file (main.tex). After each part is defined, the appropiate subfile will be included/inputed. Within that file 
	

	
	\part{Calculus}
	\chapter{Derivatives}

This is a chapter on Derivatives.

\section{Rates of Change}

This is the section on Rates of Change.

	
	\part{Abstract Algebra}
	% Abstract Algebra Book
% This is used to contain all the chapters of this book. 

\chapter{Groups}

This is the chapter on Groups.

\section{Introduction - Groups}
	
	\part{Combinatorics}
	% Combinatorics Book
% This is used to contain all the chapters of this book. 

\chapter{Graphs}

This is the chapter on Graphs.

\section{Introduction - Graphs}

This is a section on the introduction of Graphs.
		
\end{document}
